\def\line{-\hspace*{-.7zw}-}

\begin{thebibliography}{99}

\bibitem{1}
総務省. 平成26年通信利用動向調査.

\url{http://www.soumu.go.jp/johotsusintokei/statistics/data/150717_1.pdf}
(2015年12月21日アクセス).

\bibitem{2}
Wikipedia. Objective-C.

\url{https://en.wikipedia.org/wiki/Objective-C}
(2016年1月16日アクセス).

\bibitem{3}
Apple Developer. Objective-C プログラミング言語.

\url{https://developer.apple.com/jp/documentation/ObjC.pdf}
(2016年1月16日アクセス).

\bibitem{4}
Apple Developer. Xcode.

\url{https://developer.apple.com/support/xcode/jp/}
(2016年1月16日アクセス).

\bibitem{5}
Wikipedia. OpenCV.

\url{https://en.wikipedia.org/wiki/OpenCV}
(2016年1月18日アクセス)

\bibitem{6}
Build ENSIDER. OpenCVとは?最新3.0の新機能概要とモジュール構成.

\url{http://www.buildinsider.net/small/opencv/001}
(2016年1月18日アクセス).

\bibitem{7}
Apple Inc. OS X 連携.

\url{http://www.apple.com/jp/osx/continuity/}
(2016年1月20日アクセス).

\bibitem{8}
Wikipedia. Android.

\url{https://en.wikipedia.org/wiki/Android_(operating_system)}
(2016年1月22日アクセス).

\bibitem{9}
All About デジタル. Androidとは何かをわかりやすく解説!.

\url{http://allabout.co.jp/gm/gc/3588/}
(2016年1月22日アクセス).

\bibitem{10}
Wikipedia. Microsoft Windows.

\url{https://en.wikipedia.org/wiki/Microsoft_Windows}
(2016年1月22日アクセス).

\bibitem{11}
マイナビニュース. 依然として1割超えるWindows XP - 7月OSシェア

\url{http://news.mynavi.jp/news/2015/08/06/144/}
(2016年1月22日アクセス).

\bibitem{12}
Microsoft Corporation. Windowsの歴史.

\url{http://windows.microsoft.com/ja-jp/windows/history#T1=era0}
(2016年1月23日アクセス).

\bibitem{13}
List of Microsoft Windows versions.

\url{https://en.wikipedia.org/wiki/List_of_Microsoft_Windows_versions}
(2016年1月23日アクセス).

\bibitem{14}
NETMARKETSHARE. Mobile/Tablet Operating System Market Share.

\url{http://marketshare.hitslink.com/operating-system-market-share.aspx?qprid=8&qpcustomd=1}
(2016年1月24日アクセス).

\bibitem{15}
tesseract-ocr. README.

\url{https://github.com/tesseract-ocr/tesseract/blob/master/README.md}
(2016年1月27日アクセス).

\bibitem{16}
Wikipedia. Model View Controller.

\url{https://ja.wikipedia.org/wiki/Model_View_Controller}
(2016年1月28日アクセス).

\bibitem{17}
RubyLife. RailsにおけるMVC(モデル/ビュー/コントローラ).

\url{http://www.rubylife.jp/rails/ini/index7.html}
(2016年1月28日アクセス).

\bibitem{18}
CodeZine. PaaSの基礎知識とHerokuで開発を始める準備.

\url{http://codezine.jp/article/detail/8051}
(2016年2月1日アクセス).

\bibitem{19}
IT用語辞典. RDBMS.

\url{http://e-words.jp/w/RDBMS.html}
(2016年2月1日アクセス).

\bibitem{20}
Wikipedia. CentOS.

\url{https://en.wikipedia.org/wiki/CentOS}
(2016年2月1月アクセス).

\bibitem{21}
ZDNet Japan. Windowsの変遷を画像で振り返る--「Windows 1.0」から「10」まで.

\url{http://japan.zdnet.com/article/35067916/}
(2016年2月2日アクセス).

\end{thebibliography}
