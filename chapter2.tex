\chapter{技術概要}
\label{chap:concept}

\section{iOSアプリケーション}
\subsection{Objective-C}
Objective-cは, C言語をベースとして開発されたプログラミング言語である.
NEXTSTEP\footnote{NeXTコンピュータに搭載されたオブジェクト指向マルチタスクオペレーションシステム.}とMac OS X\footnote{Machintoshコンピュータに搭載されているオペレーティング・システム.}に標準付属されており, 現在は主にMac OS XとiOS上で動作するアプリケーションを開発する際に利用されている.
Objective-CはGCCによってコンパイル可能であるため, 基本的にはUNIX, Linix系OSや, GCCコンパイラを使用できる環境であればプログラムを動作させることが可能である.

C言語にSmalltalkをマクロ的拡張として施した言語であり, C言語により近い拡張を施されたC++と異なり, C言語とオブジェクト指向が混在した言語であると述べるほうが適切である.

if/for/whileなどの制御文や, int/char/floatなどのスカラー型, 関数記法, 宣言や代入といった基本的な文法はC言語に準拠しているが, オブジェクト指向はSmalltalkの概念を借用したものであり, その大きな特徴である“メッセージパッシングによるメソッド呼び出しを利用した, 記述力に優れたオブジェクトシステム”を継承している.

\subsubsection{2.1.1.1 Objective-Cの特徴}
Objective-Cによるオブジェクト指向開発環境は, 以下の3要素から成り立つ.
\begin{itemize}
\item オブジェクト

オブジェクト指向言語を利用したプログラムは, オブジェクトと呼ばれる要素を中心に構成される.
オブジェクトとは, データと, そのデータを利用したりデータに作用する特定の関数による操作を関連付けたものをいい, これらの操作はオブジェクトのメソッドと呼ばれる.
メソッドが作用するデータをインスタンス変数という.
よって, オブジェクトはインスタンス変数とメソッドを, 自己完結型のプログラミング単位にまとめたものと言い換えることができる.

Objective-Cでは, オブジェクトのインスタンス変数はオブジェクト内に存在し, 一般には, オブジェクトのメソッドによってのみ作用される.
他のメソッドがインスタンス変数に格納されているデータを得るには, アクセスしたいインスタンス変数を内包しているオブジェクトが, データを提供するメソッドを実装していなければならないが, インスタンス変数の有効範囲を指定することで, サブクラスや他のオブジェクトからインスタンス変数に直接アクセスすることも可能である.

\item 開発ツールスイート

Objective-Cで開発を行うための開発ツールとして代表的なものに, Xcodeが挙げられる.
XcodeはApple Inc.が開発, 配布している, Mac/iPhone/iPad/Apple Watch向けアプリケーション開発用のデベロッパツールセットである.
ユーザインターフェースのデザイン, コーディング, テスト, デバッグ, Apple Storeへの提出を行えることによる, スムーズなアプリケーション開発を実現している.

また, Xcodeには, Xcode IDE(統合開発環境:Integrated Development Environment), SwiftおよびObjective-Cのプログラミング言語向けコンパイラ, Instruments分析ツール, iOS Simukator, 最新のiOS/OS X向けSDKなどの機能も用意されている.

\item ランタイム環境

Objective-Cでは, 可能な限り多くの決定が実行時に行われる.
オブジェクトの作成やどのメソッドを呼び出すかの決定などの動作は動的に実行されるため, Objective-Cでは, コンパイラだけでなく, コンパイルしたコードを実行するランタイムシステムも必要となる.
ランタイムシステムは, Objective-Cにとって, 一種のオペレーティングシステムとして動作し, 言語を機能させる.
\end{itemize}

\subsubsection{2.1.1.2 Objective-C 2.0}
Apple Inc.はMac OS X v10.5 Leopardにおいて, 言語仕様を変更したObjective-C 2.0を発表した.
Objective-C 2.0では, ガベージコレクションとプロパティの導入, foreach文の採用, 実装オプションのプロトコル定義の増加, 非公開プライベートメソッドが実装可能, ランタイム構造の変更などが行われた.

\subsection{OpenCV}
OpenCV(Open Source Computer Vision Library)とは, Intel Corporationが開発, 公開しているオープンソースのコンピュータビジョン向けライブラリである.
画像処理, 画像解析, 機械学習の機能を持つC/C++, Java, Python, MATLAB用ライブラリで, プラットフォームとしてUnix/Linux系OS, Windows, Android, iOSなどをサポートしている.
BSDライセンスで配布されているため, 学術目的のみでなく, 商用目的でも利用することが可能である.

1999年に開発プロジェクトが開始され, 2000年のアルファ版公開を皮切りに, 2001年から2005年にかけて, 5つのベータ版が公開されている.
正式版のリリース後, 2016年1月現在まで開発, 公開されている.(表2.1参照.)

\begin{table}[htb]
\begin{center}
\begin{tabular}{|c|c|c|} \hline
バージョン & リリース年月 & 公式サポート言語 \\ \hline \hline
1.0 & 2006年10月 & C言語 \\ \hline
2.0 & 2009年10月 & C言語, C++ \\ \hline
2.1 & 2010年4月 & C言語, C++ \\ \hline
2.2 & 2010年12月 & C言語, C++, Python \\ \hline
2.3 & 2011年7月 & C言語, C++, Python \\ \hline
2.4 & 2012年5月 & C言語, C++, Python \\ \hline
2.4.1 & 2012年6月 & C言語, C++, Python \\ \hline
2.4.2 & 2012年7月 & C言語, C++, Python \\ \hline
2.4.3 & 2012年11月 & C言語, C++, Python \\ \hline
2.4.4 & 2013年3月 & C言語, C++, Python, Java \\ \hline
2.4.5 & 2013年4月 & C言語, C++, Python, Java \\ \hline
2.4.6 & 2013年7月 & C言語, C++, Python, Java \\ \hline
2.4.7 & 2013年11月 & C言語, C++, Python, Java \\ \hline
2.4.8 & 2013年12月 & C言語, C++, Python, Java \\ \hline
2.4.9 & 2014年4月 & C言語, C++, Python, Java \\ \hline
2.4.10 & 2014年10月 & C言語, C++, Python, Java \\ \hline
2.4.11 & 2015年2月 & C言語, C++, Python, Java \\ \hline
3.0 & 2015年6月 & C++, Python, Java(3.0以降, C言語はメンテナンス対象外) \\ \hline
3.1 & 2015年12月 & C++, Python, Java \\ \hline
\end{tabular}
\caption{OpenCVのバージョンとリリース年月, 公式サポート言語の一覧}
\end{center}
\end{table}

最新バージョンであるOpenCV 3.x系列では, C言語関数形式のインターフェイスがメンテナンス対象外とされ, C++ APIを利用することが推奨されている.

\subsubsection{2.1.2.1 OpenCVの機能}
OpenCVに実装されている機能としては, フィルター処理, 変形処理, 構造解析と形状ディスクリプタ, 物体検出, モーション解析と物体追跡, 領域分割, カメラキャリブレーション, 特徴点検出, 機械学習がある.

\begin{itemize}
\item フィルター処理

OpenCVにおけるフィルタリングは, 二次元配列であるMat型で定義された二次元画像に対して処理を実行する.
画像の平滑化や収縮/膨張などの処理を行うことで, 画像のぼかしやノイズキャンセリングを施す.

\item 変形処理

二次元画像の幾何学変換を行い, 変形画像をマッピングする.
この処理を施す際は画像の内容そのものは変更されず, ピクセルの座標移動のみを行う..

\item 構造解析と形状ディスクリプタ

二次元画像内に描かれている, あるいは含まれている特定の形状を検出する.
画像内の特定の形状のみを認識したい場合等で利用する.

\item 物体検出

予め設定しているテンプレートと二次元画像とを比較し, テンプレートとマッチングする物体を検出する.
顔認識や指紋認識など, 生体認証等で利用する.

\item モーション解析と物体追跡

動画内で動いている物体の動きを認識する, あるいは動体を追跡する.
モーション検知や動体検知等で利用する.

\item 領域分割

二次元画像を形状や色などの特徴を用いて, 複数の領域に分割する.
特定の領域のみで画像処理を施す場合や, 特定の領域のみを抜き出す場合等で利用する.

\item カメラキャリブレーション

カメラ固有の内部パラメータと, ワールド座標系における位置や姿勢を意味する外部パラメータを求める.
カメラで撮影した二次元画像の歪みを補正する場合等で利用する.

\item 特徴点検出

二次元画像における特徴点を検出する.
特徴点を検出してパターンマッチングを行う場合等で利用する.

\item 機械学習

OpenCVでは, ニューラルネットワークを用いた機械学習を実装することが可能である.
二次元画像を教師データとして入力し, 形状や物体の認識率を向上させる場合等で利用する.
\end{itemize}

\subsection{Tesseract-OCR}
Tesseract-OCRとは, ヒューレット・パッカード研究所とヒューレット・パッカード社が開発し, 現在はGoogleが開発, 配布している, 光学文字認識用のオープンソースライブラリである.

ライブラリの大半はC言語で書かれており, 一部はC++で書かれている.
そのため, C言語コンパイラがインストールされているならば, 様々な環境で動作させる事ができる.

Tesseract-OCRは2016年1月現在, 63の言語に対応している.(表2.2参照.)

\begin{table}[htb]
\begin{center}
\begin{tabular}{|p{3.5cm}|p{3.5cm}|p{3.5cm}|p{3.5cm}|} \hline
英語 & ウクライナ語 & トルコ語 & タイ語 \\ \hline
タガログ語 & テルグ語 & タミル語 & スウェーデン語 \\ \hline
スワヒリ語 & セルビア語 & アルバニア語 & スペイン古語 \\ \hline
スペイン語 & スロベニア語 & スロバキア語 & ローマ語 \\ \hline
ポルトガル語 & ポーランド語 & ノルウェー語 & オランダ語 \\ \hline
マレー語 & マルタ語 & マケドニア語 & マラヤーラム語 \\ \hline
リトアニア語 & ラトビア語 & 韓国語 & カナダ語 \\ \hline
イタリア古語 & イタリア語 & アイスランド語 & インドネシア語 \\ \hline
チェロキー語 & ハンガリー語 & クロアチア語 & ヒンディー語 \\ \hline
ヘブライ語 & ガルシア語 & 中世フランス語 & フランス語 \\ \hline
フランク語 & フィン語 & バスク語 & エストニア語 \\ \hline
エスペラント語 & 中世英語 & グリーク語 & ドイツ語 \\ \hline
デンマーク語 & チェコ語 & カタロニア語 & ブルガリア語 \\ \hline
ブルガリア語 & ベンガル語 & ベラルーシ語 & アゼルバイジャン語 \\ \hline
アラビア語 & アフリカ語 & 日本語 & 中国語(簡体字) \\ \hline
中国語(繁体字) & ロシア語 & ベトナム語 & 数学記号 \\ \hline
\end{tabular}
\caption{Tesseract-OCRにて認識可能な言語一覧}
\end{center}
\end{table}

Tesseract-OCRはニューロンネットワークによる機械学習を実装しており, 上記の言語のデータを教師データとして与えることにより, 文字の認識率を向上させることが可能である.

\section{Webアプリケーション}
\subsection{Ruby on Rails}
Ruby on Railsとは, Rails Core Teamが開発, 提供しているオープンソースのWebアプリケーションフレームワークである.

2004年10月に正式版がリリースされてから, 2016年1月現在まで開発, 公開されている.(表2.3参照.)

\begin{table}[htb]
\begin{center}
\begin{tabular}{|l|l|} \hline
バージョン & リリース年月 \\ \hline \hline
0.8.0 & 2004年10月 \\ \hline
0.9.0 & 2004年12月 \\ \hline
0.10.0 & 2005年2月 \\ \hline
0.11.0 & 2005年3月 \\ \hline
0.12.0 & 2005年4月 \\ \hline
0.13.0 & 2005年7月 \\ \hline
0.14.1 & 2005年10月 \\ \hline
1.0.0 & 2005年12月 \\ \hline
1.1.0 & 2006年3月 \\ \hline
1.2.0 & 2007年1月 \\ \hline
2.0.0 & 2007年12月 \\ \hline
2.1.0 & 2008年5月 \\ \hline
2.2.2 & 2009年11月 \\ \hline
2.3.2 & 2009年5月 \\ \hline
3.0.0 & 2010年8月 \\ \hline
3.1.0 & 2011年8月 \\ \hline
3.2.0 & 2012年1月 \\ \hline
4.0.0 & 2013年6月 \\ \hline
4.1.0 & 2014年4月 \\ \hline
4.2.0 & 2014年12月 \\ \hline
4.2.5 & 2015年11月 \\ \hline
\end{tabular}
\caption{Ruby on Railsのメジャーバージョンとリリース年月の一覧}
\end{center}
\end{table}

\subsubsection{2.2.1.1 Ruby on Railsの基本理念}
Ruby n Railsの特徴は, Ruby on Railsの開発理念である“同じことを繰り返さない(DRY:Don't Repeat Yourself)”と'“設定より規約(CoC:Convention over Configuration)”の順守による, 実アプリケーションの開発が他のフレームワークよりも少ないコードで簡単に行える点である.

“同じことを繰り返さない”とは, 「定義などの作業は一度のみ行う」, 「重複するコードを記述しない」という意味である.

“設定より規約”とは, 「慎重に設定された規約に従うことで, 設定を不要にする, あるいは軽減する」という意味である.

これらの基本理念により, Ruby on Railsでは, コンソール上でコマンドを入力することで, 命名規則に従った複数の関連付けられたファイルが生成されるため, ファイル構成の把握が非常に簡易である.

このRuby on Railsのコンセプトに影響を受けたフレームワークとして, PHPのCakePHPやSymfony, PerlのCatalyst, JavaのGrailsといったものが挙げられる.

\subsubsection{2.2.1.2 Ruby on RailsにおけるMVCアーキテクチャ}
Ruby on RailsはMVC(Model View Controller)アーキテクチャを採用している.

MVCアーキテクチャとは, アプリケーションの内部データをユーザが直接参照することがないよう, Model, View, Controllerの3つの要素に分割したものである.

\begin{itemize}
\item Model

アプリケーションが扱う領域のデータと手続きを表現した要素.
データの変更をビューに通知するのもModelの役割である.

Ruby on RailsにおけるModelは, 使用しているデータベースのテーブルごとにModelが用意されている.
利用者からのリクエストで呼びだされたアクションは, Modelを介してデータベースとのやり取りを行い, データの取得や新しいデータの格納を行う.

\item View

Modelのデータを取り出し, ユーザが見るのに適した形で表示する要素.
すなわち, データをUIへ出力するのが役割である.

Ruby on RailsにおけるViewは, アプリケーション内に複数用意されている.
Viewの1つ1つが与えられたデータから生成されたHTML文書となっている.
アクションに対応するViewが1つ用意されており, アクションが実行されると, 紐付けされたViewが呼び出されて利用者へ返すWebページを作成する.

\item Controller

ユーザからの入力をModelへのメッセージへと変換してModelへと伝える要素である.
すなわち, UIからの入力を担当する.

Ruby on RailsにおけるControllerは, アプリケーション内に複数用意されている.
1つ1つのControllerは, ユーザの入力をアクションとして紐付けされているModelへと命令する.

\end{itemize}

\subsection{PostgreSQL}
PostgreSQLとは, PostgreSQL Global Development Groupがオープンソースで開発, 配布している, オブジェクト関係データベース管理システム(ORDBMS:Objective Relational DataBase Management System)である.

1995年5月に正式版がリリースされてから, 2016年1月現在まで開発, 公開されている.(表2.4参照.)

\begin{table}[htb]
\begin{center}
\begin{tabular}{|l|l|} \hline
バージョン & リリース年月 \\ \hline \hline
0.01 & 1995年5月 \\ \hline
1.0 & 1995年9月 \\ \hline
6.0 & 1997年1月 \\ \hline
6.1 & 1997年6月 \\ \hline
6.2 & 1997年10月 \\ \hline
6.3 & 1998年3月 \\ \hline
6.4 & 1998年10月 \\ \hline
6.5 & 1999年6月 \\ \hline
7.0 & 2000年5月 \\ \hline
7.1 & 2001年4月 \\ \hline
7.2 & 2002年2月 \\ \hline
7.2 & 2002年11月 \\ \hline
7.4 & 2003年11月 \\ \hline
8.0 & 2005年1月 \\ \hline
8.1 & 2005年11月 \\ \hline
8.2 & 2006年12月 \\ \hline
8.3 & 2008年2月 \\ \hline
8.4 & 2009年7月 \\ \hline
9.0 & 2010年9月 \\ \hline
9.1 & 2011年9月 \\ \hline
9.2 & 2012年9月 \\ \hline
9.3 & 2013年9月 \\ \hline
9.4 & 2014年12月 \\ \hline
9.5 & 2016年1月 \\ \hline
\end{tabular}
\caption{PostgreSQLのメジャーバージョンとリリース年月の一覧}
\end{center}
\end{table}

PostgreSQLは, オープンソースなリレーショナルデータベース管理システム(RDBMS)であるIngresから発展したプロジェクトである.
PostgreSQLという名は, 「Post-Ingres」, つまりIngresの後継であることと, SQLデータベース操作構文を用いてデータベース操作が可能であることが, その名の由来となっている.
Ingresの課題であった, 「ユーザが新たな定義域を既存の単純な定義域から定義できない」という実装の限界に対処することを目的として開発された.

データベースの操作にはSQLデータベース操作構文を利用しているが, PL/pgSQL, PL/PSM, スクリプト言語(PL/Perl, PL/php, PL/Python, PL/Ruby, PL/Tcl, PL/Lua), コンパイラ言語(C言語, PL/Java), 統計処理言語(PL/R)の関数を実行することも可能である.

\subsubsection{2.2.2.1 リレーショナルデータベース(Relational DataBase)}
リレーショナルデータベースとは, 一見のデータを複数の属性(カラム)の値(フィールド)の組(レコード)として表現し, 組を列挙することでデータを格納していく方式のデータベースのことである.
属性を列, 組を業とする表(テーブル)の形で表されることが多い.
複数のテーブルに含まれる同じカラムを関連付けることができ, 複雑なデータや大規模なデータを柔軟に取り扱うことができる.

データベースの構造では最も普及しており, 単にデータベースといった場合はリレーショナルデータベースであることが多い.

\subsubsection{2.2.2.2 Relational DataBase Management System}
Relational DataBase Management System(RDBMS)とは, リレーショナルデータベースを管理するための専用ソフトウェアである.

ストレージ内に専用の管理領域を確保し, テーブルの作成や消去, 構造の修正, データの追加, 検索, 抽出, 修正, 削除等を行う.
データベース管理者や利用者が直接操作を行う他に, 外部のソフトウェアから接続を受け付け, ソフトウェアのプログラムから操作を行うことも可能である.
RDBMSへの照会や操作の指示には, SQLと呼ばれるデータベース操作言語が標準として広く利用されている.

RDBMSは, 不正なデータの記録を拒否するなどしてデータベースの整合性を保つ, 権限のない利用者による不正な読み出しや改ざん等からデータを保護する仕組みを持つ.
また, 関連する複数の処理を一体化して矛盾なく実行するトランザクション処理を行ったり, 障害に備えてデータベースのバックアップを行い, 破損時には過去のある時点の状態へと復旧したりといった機能を備えたものもある.

\subsection{Heroku}
Herokuとは, 2007年に創業された同名の企業が開発と運営を行っている, PaaS(Platform as a Service)である.

今回は, Ruby on Railsで作成したWebアプリケーションをHerokuサーバにて動作させ, インターネットに公開するために利用する.

PaaSとは, 「アプリケーションを実行するためのプラットフォーム」であり, プラットフォームのうち, アプリケーションとデータのみを自社管理し, ランタイム/OS/仮想化/物理サーバ/ストレージ/ネットワークはサービス業者が管理する.

\section{VPSサーバ}
\subsection{CentOS}
CentOSは, The CentOS Projectがオープンソースライセンスで開発, 配布しているLinixディストリビューションである.
CentOSという名称は「コミュニティベースで開発された, 有償版に匹敵するオペレーティングシステム(Community ENTerprise Operating System)」が由来となっている.

2004年5月のリリースから, RHELの最新版が公開される度に, その後を追うようにしてメジャーアップデートを重ね, 2016年1月現在まで開発, 配布されている.

Red Hat Enterprise Linux(RHEL)\footnote{Red Hat社が開発, 販売している業務向けLinuxディストリビューション.}の完全互換を目指して開発され, Red Hat社がオープンソースにて公開しているRHELのソースコードを基に, Red Hat社の商標や商用パッケージを除去し, リビルドしたものがCentOSである.
このことから, CentOSは一般に“RHELクローン”\footnote{RHELを基としたLinuxディストリビューション全般を含んだ呼称であるため, White Box Enterprise LinuxやScientific LinuxもRHELクローンに含まれる.}と呼称されることもある.

Linuxをベースとしたオペレーティングシステムであること, GNUライセンスによる配布による低コストでの導入の容易さから, 主に中小規模のプロジェクトにて採用されることが多い.
