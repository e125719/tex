\chapter{システムの実装}
\label{chap:poordirection}

\section{車両認識システム}
本研究にて構築するシステムは, 主に「iOSアプリケーション」, 「Webアプリケーション」, 「データベース」の三要素にて成り立つ(図3.1).

\subsection{システムの流れ}
本研究にて構築するシステムは,
\begin{screen}
\begin{itembox}[l]{iOSアプリケーション}
\begin{enumerate}
\item iOSアプリケーションにて車両のナンバープレートを撮影する.
\item iOSアプリケーションの内部処理にて, 撮影画像から車両ナンバーを文字列として取得する.
\item 取得した車両ナンバーをWebアプリケーション側のPostgreSQLに送信する.
\end{enumerate}
\end{itembox}

\begin{itembox}[l]{Webアプリケーション}
\begin{enumerate}
\item Webアプリケーションにログインする.
\item PostgreSQLデータベースに各種操作を行う.
\item Webアプリケーションからログアウトする.
\end{enumerate}
\end{itembox}
\end{screen}
という流れを想定して開発した.

\begin{figure}
\begin{center}
\includegraphics[width=16cm]{fig/system.pdf}
\end{center}
\caption{車両ナンバー認識システムの簡易システム図}
\end{figure}

\begin{figure}
\begin{center}
\includegraphics[width=16cm]{fig/usecase_system.png}
\end{center}
\caption{車両ナンバー認識システムのユースケース図}
\end{figure}

\section{iOSアプリケーション}
iOSアプリケーションを作成するにあたり, 以下の言語とライブラリ, 統合開発環境を利用した(表3.1).

\begin{table}
\begin{center}
\begin{tabular}{|l|l|} \hline
Xcode &  \\ \hline
Objective-C 2.0 & \\ \hline
Tesseract-OCR & \\ \hline
OpenCV & \\ \hline
\end{tabular}
\end{center}
\caption{iOSアプリケーション作成時に利用した環境のバージョン一覧}
\end{table}

\subsection{要件定義}
iOSアプリケーションには, 以下の機能を実装する.
\begin{screen}
\begin{itemize}
\item iOSデバイスのカメラにて撮影する
\item 撮影画像から目的の文字列を取得する
\item 取得した文字列をWebアプリケーションのデータベースにPOSTする.
\end{itemize}
\end{screen}

これらの要件を, UMLを用いて分析する.

\subsection{ユースケース}
ユーザがiOSデバイスのカメラを用いて撮影した画像から, 文字列を検出する.
その文字列をWebアプリケーションのデータベースに送信し, 集積する(図3.3).

\begin{figure}
\begin{center}
\includegraphics[width=16cm]{fig/usecase_ios.png}
\end{center}
\caption{iOSアプリケーションのユースケース図}
\end{figure}

\subsection{クラス}
\begin{figure}
\begin{center}
\includegraphics[width=16cm]{fig/class_ios.png}
\end{center}
\caption{iOSアプリケーションのクラス図}
\end{figure}

\begin{description}

\item (A) Cameraクラス

Cameraクラスでは, 「iOSアプリケーションのMainViewにボタン等を配置する」, 「iOSデバイスのカメラを起動し, 撮影した画像を保存する」, 「他クラスのメソッドを呼び出す」ことが可能である.

\begin{itemize}
\item 属性

\begin{breakbox}
\begin{itemize}
\item imageView:UIImageView

撮影した画像を出力するためのローカル変数.

\item image:UIImage

撮影した画像.

\item procImage:UIImage

Processingクラスにて画像処理された画像.

\item textField:UITextField

Recognizeクラスにて取得した文字列を表示するためのテキストボックス.

\item result:NSString

Recognizeクラスにて取得してきた文字列.
\end{itemize}
\end{breakbox}

\item 操作

\begin{itembox}[l]{makeButton:UIButton}
iOSアプリケーションのMainViewにボタンを作成するメソッド.

ボタンを配置する座標, サイズ, ボタンに書かれる文字列, ボタンに付加するタグを指定して呼び出すことで, iOSアプリケーションのMainViewにボタンを作成し配置する.

\begin{itembox}[l]{引数}
\begin{itemize}
\item rect:CGRect

ボタンの座標, サイズを決定する変数.

\item text:NSString

ボタンに書かれる文字列を決定する変数.

\item tag:int

ボタンに整数型のタグを付加するための変数.
\end{itemize}
\end{itembox}
\end{itembox}

\begin{itembox}[l]{makeImage:UIImageView}
iOSアプリケーションのMainViewに画像を表示するためのSubViewを作成するメソッド.

画像を配置する座標, サイズ, 出力する画像を指定して呼び出すことで, iOSアプリケーションのMainViewに画像を出力する.

\begin{itembox}[l]{引数}
\begin{itemize}
\item rect:CGRect

画像を出力するSubViewの座標, サイズを指定する変数.

\item image:UIImage

SubViewに出力する画像を指定する変数.
\end{itemize}
\end{itembox}
\end{itembox}

\begin{itembox}[l]{makeTextField:UITextField}
iOSアプリケーションのMainViewにテキストフィールドを作成するメソッド.

テキストフィールドを配置する座標, サイズ, 入力されるテキストを指定して呼び出すことで, iOSアプリケーションのMainViewにテキストフィールドを作成し配置する.

\begin{itembox}[l]{引数}
\begin{itemize}
\item rect:CGRect

テキストフィールドを配置する座標, サイズを指定する変数.

\item text:NSString

テキストフィールドに入力する文字列を指定する変数.
\end{itemize}
\end{itembox}
\end{itembox}

\begin{itembox}[l]{textFieldShouldReturn:BOOL}
テキストフィールドにて, キーボードでリターンキーを押した際の挙動を制御するメソッド.

文字列を編集後にリターンキーを押した際に呼び出され, SubViewとして呼びだされているキーボードを引っ込める.

\begin{itembox}[l]{引数}
\begin{itemize}
\item textField:UITextField

編集したい文字列が入力されているテキストフィールド.
\end{itemize}
\end{itembox}
\end{itembox}

\begin{itembox}[l]{viewDidLoad:void}
アプリケーションが起動した際に一度だけ読み込まれるメソッド.

変数の初期化やMainViewにボタン等を配置する際に利用する.
\end{itembox}

\begin{itembox}[l]{didReceiveMemoryWarning:void}
iOSアプリケーション使用時に, メモリが不足する際に呼び出されるメソッド.

すべてのViewに対して参照を開放する際に利用する.
\end{itembox}

\begin{itembox}[l]{showAlert:void}
アラートビューを表示するメソッド.

アラートビューのタイトルと表示するアラートの内容を指定して呼び出すことで, iOSアプリケーションのMainView上にSubViewとしてアラートが表示される.

\begin{itembox}[l]{引数}
\begin{itemize}
\item title:NSString

アラートビューのタイトルを指定する変数.

\item text:NSString

アラートビューに表示するテキストを指定する変数.
\end{itemize}
\end{itembox}
\end{itembox}

\begin{itembox}[l]{openPicker:void}
iOSアプリケーションにて, 画像を取得するためのメソッド.

カメラ, フォトアルバム, フォトライブラリから画像を取得する.

このメソッドでは, iOSのカメラ機能にて撮影した画像を取得する際に呼び出す.

\begin{itembox}[l]{引数}
\begin{itemize}
\item sourceType:UIImagePickerSourceType

イメージピッカーの属性を決定する変数.
カメラ, フォトアルバム, フォトライブラリのいずれかから選択する.
\end{itemize}
\end{itembox}
\end{itembox}

\begin{itembox}[l]{imagePickercontroller:void}
画像取得後に呼び出されるメソッド.

画像を取得した後の処理を行う他クラスのメソッドを呼び出す.

\begin{itembox}[l]{引数}
\begin{itemize}
\item picker:UIImagePickerController

画像取得元のイメージピッカーを指定する変数.

\item info:NSDictionary$<$NSString *, id$>$

取得した画像の情報を格納する変数.
\end{itemize}
\end{itembox}
\end{itembox}

\begin{itembox}[l]{clickButton:IBAction}
iOSアプリケーション上のMainViewに配置されているボタンを押下した際に呼び出されるメソッド.

\begin{itembox}[l]{引数}
\begin{itemize}
\item sender:UIButton

どのボタンが押下されたのかを決定する変数.
\end{itemize}
\end{itembox}
\end{itembox}

\end{itemize}


\item (B) Processingクラス

Processingクラスでは, 「撮影画像をOpenCVにて加工可能となるように処理を施す」, 「撮影画像中の文字列が含まれている箇所に注目する」, 「注目している箇所のみで光学的文字認識が施されるように, 撮影画像を加工する」ことが可能である.

\begin{itemize}
\item 属性

\begin{breakbox}
\begin{itemize}
\item \_src:Mat

入力画像を指定する変数.

\item \_gray:Mat

グレースケール変換された画像を格納する変数.

\item binRoi:Mat

バイナリスケール変換された画像を格納する変数.

\item corners:vector$<$KeyPoint$>$

検出した特徴点を格納する変数.

\item rect:Rect

特徴点を内包した矩形の座標, サイズ決定する変数.

\end{itemize}
\end{breakbox}

\newpage

\item 操作

\begin{itembox}[l]{cvtMat:Mat}
UIImage型の画像をMat型に変換するメソッド.

UIImage型の入力画像を指定して呼び出すことで, OpenCVで処理可能なMat型に変換する.

\begin{itembox}[l]{srcImage:UIImage}
\begin{itemize}
\item srcImage:UIImage

UIImage型の画像を格納する変数.
\end{itemize}
\end{itembox}
\end{itembox}

\begin{itembox}[l]{cvtGrayMat:Mat}
Mat型画像をグレースケール化するメソッド.

cvtMatにてMat型に変換した入力画像にグレースケール化処理を施す.

\begin{itembox}[l]{引数}
\begin{itemize}
\item srcMat:Mat

3チャネルの入力カラー画像を格納する変数.
\end{itemize}
\end{itembox}
\end{itembox}

\begin{itembox}[l]{findCorner:vector$<$KeyPoint$>$}
グレースケール画像から特徴点を検出するメソッド.

cvtGrayScaleにてグレースケール処理を施した画像を指定することで, その画像中から特徴点を検出する.

\begin{itembox}[l]{引数}
\begin{itemize}
\item srcMat:Mat

グレースケール画像を格納する変数.
\end{itemize}
\end{itembox}
\end{itembox}

\begin{itembox}[l]{makeRect:Rect}
特徴点を内包する矩形の座標, サイズを決定するメソッド.

特徴点の集合を指定することで, 特徴点の多い部分に注目するための矩形を設定する.

\begin{itembox}[l]{引数}
\begin{itemize}
\item kp:vector$<$KeyPoint$>$

特徴点の集合を格納する変数.

\item srcMat:Mat

グレースケール画像を格納するための変数.
\end{itemize}
\end{itembox}
\end{itembox}

\begin{itembox}[l]{cvyBinaryRoi:Mat}
Mat型カラー画像をバイナリ画像へ変換するメソッド.

makeRectにて設定した矩形を指定することで, 注目している範囲をバイナリ画像化する.

\begin{itembox}[l]{引数}
\begin{itemize}
\item srcMat:Mat

3チャネルの入力カラー画像を格納する変数.

\item roiRect:Rect

注目する矩形範囲を決定する変数.
\end{itemize}
\end{itembox}
\end{itembox}

\begin{itembox}[l]{provessing:UIImage}
Processingクラスのメソッドをコントロールするメソッド.

\begin{itembox}[l]{引数}
\begin{itemize}
\item srcImage:UIImage

CameraクラスのopenPickerControllerメソッドにて取得した撮影画像を格納する変数.
\end{itemize}
\end{itembox}
\end{itembox}
\end{itemize}

\newpage

\item (C) Recognizeクラス

Recognizeクラスでは, 「光学的文字認識を実行する」, 「取得した文字列をCameraクラスに渡す」ことが可能である.

\begin{itemize}
\item 属性

\begin{breakbox}
\begin{itemize}
\item \_origImage:UIImage

光学的文字認識を実行する画像を格納するための変数.
\end{itemize}
\end{breakbox}

\item 操作

\begin{itembox}[l]{setOrigImage:void}
光学的文字認識を実行する画像を取得するためのメソッド.

Processingクラスのprocessingメソッドにて処理の施された画像を受け取り, \_origImageに格納する.

\begin{itembox}[l]{引数}
\begin{itemize}
\item image:UIImage

光学的文字認識を施す画像を指定する変数.
\end{itemize}
\end{itembox}
\end{itembox}

\begin{itembox}[l]{analyze:void}
Tesseract-OCRによる光学的文字認識を実行するメソッド.
\end{itembox}
\end{itemize}

\item (D) PostDBクラス
\begin{itemize}
\item 属性

\begin{breakbox}
\begin{itemize}
\item \_conn:PGconn

PostgreSQLデータベースとの接続状態を格納する構造体.

\item \_conninfo:const char

PostgreSQLデータベースの接続する際に必要となる情報を格納する変数.

\item \_query:char

接続したPostgreSQLにて実行するクエリを決定する変数.
\end{itemize}
\end{breakbox}

\item 操作

\begin{itembox}[l]{connDB:void}
WebアプリケーションのPostgreSQLに接続するメソッド.

\begin{itembox}[l]{引数}
\begin{itemize}
\item conninfo:const char

PostgreSQLへの接続状態を保存する変数.
\end{itemize}
\end{itembox}
\end{itembox}

\begin{itembox}[l]{execDB:void}
WebアプリケーションのPostgreSQLデータベースへクエリを発行し, データベースへデータを追加する.

\begin{itembox}[l]{引数}
\begin{itemize}
\item str:NSString

PostgreSQLデータベースへ送信する文字列を決定する変数.
\end{itemize}
\end{itembox}
\end{itembox}

\begin{itembox}[l]{postData:void}
PostDBクラスのメソッドをコントロールするメソッド.

\begin{itembox}[l]{引数}
\begin{itemize}
\item postString:NSString

WebアプリケーションのPostgreSQLデータベースへ送信する文字列を決定する変数.
\end{itemize}
\end{itembox}
\end{itembox}

\end{itemize}

\end{description}

\section{Webアプリケーション}
Webアプリケーションを作成するにあたり, 以下の言語とライブラリ, サービスを利用した(表3.2).

\begin{table}
\begin{center}
\begin{tabular}{|l|l|} \hline
rbenv & \\ \hline
Ruby & \\ \hline
RubyGems & \\ \hline
Ruby on Rails & \\ \hline
PostgreSQL & \\ \hline
Heroku & \\ \hline
\end{tabular}
\end{center}
\caption{Webアプリケーション作成時に利用した環境のバージョン一覧}
\end{table}

\subsection{要件定義}
Webアプリケーションには, 以下の機能を実装する.

\begin{screen}
\begin{itemize}
\item データベースにて, iOSアプリケーションから送信された文字列を送信された日時/時刻とともに管理する.

\item 集積したデータを文字列, 日付にて検索する.
\end{itemize}
\end{screen}

これらの要件を, UMLを用いて分析する.

\subsection{ユースケース}
ユーザはiOSアプリケーションから送信された全データを閲覧することができ, 文字列, 送信した日付で検索することができる(図3.5).

\begin{figure}
\begin{center}
\includegraphics[width=16cm]{fig/usecase_web.png}
\end{center}
\caption{Webアプリケーションのユースケース図}
\end{figure}

\newpage

\subsection{クラス}

\begin{figure}
\begin{center}
\includegraphics[width=16cm]{fig/class_web.png}
\end{center}
\caption{Webアプリケーションのクラス図}
\end{figure}

\begin{description}

\item (A) Userクラス
\begin{itemize}
\item 属性

\begin{breakbox}
\begin{itemize}
\item current\_sign\_in\_at:datetime

Webアプリケーションにユーザ登録したユーザがサインインした時刻を記録するカラム.

\item current\_sign\_in\_ip:datetime

ユーザがサインインした際のリモートIPを記録するカラム.

\item email:string

ユーザのサインイン時に利用するEメールアドレスを記録するカラム.

\item last\_sign\_in\_at:datetime

ユーザが最終サインインした時刻を記録するカラム.

\item last\_sign\_in\_ip:datetime

ユーザが最終サインインした際のリモートIPを記録するカラム.

\item remenber\_created\_at:datetime

ユーザ登録した時刻を記録するカラム.

\item reset\_password\_sent\_at:datetime

パスワードをリセットする操作を行った際の時刻を記録するカラム.

\item reset\_password\_token:string

パスワードをリセットする操作を行った際に設定した新しいパスワードを記録するカラム.

\item sign\_in\_count:int

ユーザがサインインした回数を記録するカラム.
\end{itemize}
\end{breakbox}

\end{itemize}

\item (B) StringDBクラス
\begin{itemize}
\item 属性

\begin{screen}
\begin{itemize}
\item string:string

光学的文字認識にて取得した文字列を記録するカラム.
\end{itemize}
\end{screen}

\end{itemize}

\end{description}
