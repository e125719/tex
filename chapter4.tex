\chapter{本研究の利用例, 利用アイデア}
\label{chap:utilization}

本研究では, 「iOSのポータビリティの高さ」と「沖縄県の観光産業」に注目し, 観光地の文字情報をデータベースに集積することを行った.

しかし, 本研究は上記の目的のみにとどまらず, iOSデバイスのポータビリティを活かし, 様々な分野に派生することが可能であると考える.

\section{利用アイデア1:緊急車両や警察車両への住所通達システム}
\subsection{背景と目的}
本研究にて構築したシステムは, スマートデバイスで撮影した画像から文字列を取得し, Webアプリケーションのデータベースに送信する, というものである.

ここでは, 画像に“GPSによる位置情報”を付与する.

SNSが急速に普及して以降, 事件や事故が起きた際に, その現場周辺にいる人々がその様子を撮影し, SNSにその時の状況をアップロードする, ということが行われるようになっている.
ここでは, その“画像による状況把握の容易さ”と, “スマートデバイスにてGPSの位置情報が利用可能である”ことに注目する.

\subsection{システムの要件定義}
ユーザが現場周辺でスマートデバイスを用いて画像を撮影する.
その画像にGPSの位置情報とどの緊急車両(救急, 消防, 警察)が必要であるかの情報を付与し, Webアプリケーションに送信する.

受信したWebアプリケーションは, それらの情報を受け取ると各緊急機関に現場画像と位置情報, どの緊急車両を向かわせるか要請する.

そして, 緊急車両は指定された車両でGPSによる現場の位置情報を元に, 現場に向かう.

\subsection{このシステムを利用することで解決する問題}
このシステムを利用することで解決する問題は,
\begin{itemize}
\item 緊急車両を呼ぶ際に, 現場の住所がすぐに分からない

現場の住所を電話口で職員に伝える際, 通報者の近くに現住所を示すものがない場合, 正確な住所を伝えることが困難となることも考えられる.

このシステムでは, スマートデバイスから取得したGPSの位置情報をWebアプリケーションに送信するため, 住所が不明な場所においても, 正確な現場の位置を伝えることが可能となる.

\item 偽情報による緊急車両の出動

通報による緊急車両の出動は, 故意な偽情報によって行われることがある.
それらの悪意ある行動による緊急車両の不必要な出動は, 真に必要な現場への出動が遅延する可能性を生む.

このシステムでは, スマートデバイスのポータビリティと情報発信力を活用しており, 大規模な事件, 事故の場合は複数人が通報システムを利用することが考えられるため, 偽情報による緊急車両の出動を抑制することが可能だと推測できる.
\end{itemize}

\section{}
\subsection{}
